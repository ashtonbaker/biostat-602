\documentclass[titlepage]{article}

%\linespread{1.25}

\usepackage{../stats-homework}

\begin{document}
\title{MATH 571 Numerical Linear Algebra\\Exam 02}
\author{Ashton Baker}
\date{Thursday December 22, 2016}
\maketitle
\begin{enumerate}
\item A coin is twice as likely to turn up tails as heads. If the coin is tossed independently, what is the probability that the third head occurs on the 5th trial?

\textbf{Solution.} This implies that $\P(H) = 1/3$ and $\P(T) = 2/3$. So

\item Suppose $X$ and $Y$ are two independent variables with unit variance. Let $Z = aX + Y$, where $a > 0$. If $\cor(X, Z) = 1/3$ ,then obtain the value of $a$.

\item Let $g(x)$, $x \geq 0$ be a valid pdf for a nonnegative random variable and define
\[ f(x, y) = \frac{ g(\sqrt{x^2 + y*2}) }{ 2\pi \sqrt{x^2 + y^2} }\]
for $-\infty < x, y < \infty$.
  \begin{enumerate}
  \item Show that $f(x, y)$ is a valid pdf.

  \item Suppose that the pair $(X, Y)$ has the pdf $f(x, y)$. What is $P(XY > 0)$?
  \end{enumerate}

\item Given independent and identically distributed random samples $X_1, X_2, \ldots, X_n$, each with finite mean $\mu$ and finite variance $\sigma^2$, define
\[\begin{aligned}
\Xbar &= \frac{1}{n} \sum_{i = 1}^n X_i
S^2 &= \frac{1}{n-1} \sum_{i = 1}^n \left(X_i - \Xbar\right)^2
W^2 &= \frac{1}{n} \sum_{i = 1}^n \left(X_i - \Xbar\right)^2
\end{aligned}\]
  \begin{enumerate}
  \item Show that $S^2 \xrightarrow{P} \sigma^2$
  \item Derive the asymptotic distribution of $\frac{\sqrt{n}\left(\Xbar - \mu\right)}{\sqrt{S^2}}$
  \item Use the Delta method to derive the asymptotic distribution of $\Xbar^2$ after you normalize it appropriately.
  \end{enumerate}

\item For two sets of random varibales $\{X_i\}$, $i = 1, \ldots, n$, and $\{Y_i\}$, $j = 1, \ldots, m$, show that
\[\cov\left( \sum_{i=1}^n a_i X_i, \sum_{j=1}^m b_j Y_j = \sum_{i=1}^n \sum_{j=1}^m a_i b_j \cov(X_i, Y-j)\right)\]
where $a_i$ and $b_j$ are arbitrary constants.

\item Suppose $N \dist \poisson(\lambda)$. Given $N = n > 0$, $X_1, \ldots, X_N$ are iid and follow $\U[0, 1]$. We define $X_0 = 0$ when $N = 0$.
  \begin{enumerate}
  \item Given $N = n$, find the probability that $X_0, X_1, \ldots, X_N$ are all less that $t$, where $0 < t < 1$.

  \item Find the (unconditional) probability that $X_0, X_1, \ldots, X_N$ are all less than $t$, where $0 < t < 1$.

  \item Let $S_N = X_0 + X_1 + \cdots + X_N$. Compute $\ev(S_N)$.
  \end{enumerate}

\item Let $X_1, X_2, X_3$ be a random sample of size 3 from a $N(0, 1)$ population. In each of the following cases, $Z$ denotes a specific function derived from this random sample. In each case identify the distribution of the resulting random variable $Z$ along with the associated parameters.
  \begin{enumerate}
  \item $X_1 + X_2 + 2X_3$.

  \item $X_1^2 + X_2^2 + X_3^3$.

  \item $(X_1 - X_2)^2 / 2$.

  \item $Z = \frac{2X_1^2}{X_2^2 + X_3^2}$

  \item $Z = \frac{(X_1 - X_2)^2}{(X_1 + X_2)^2}$
  \end{enumerate}
\end{enumerate}
\end{document}
